% Detailed Design Specifications
%
% Interface specifications
% Class definitions
% Pseudocode
% Data file specifications if applicable
\subsection*{Interface Specifications}
    The classes \mintinline{python}{RenderedBase} and
    \mintinline{python}{ActorBase} will form a basis for
    the API for LazerBlast.  They describe all moving
    objects, and all actors in the game.  Among their
    most important functions, they will define a method
    to set the current action.  They will also use Python's
    magic methods to get the next frame of their actions.
    
    The class \mintinline{python}{MainMenu} is a basis for the User
    Interfaces for LazerBlast. Through it, the user can navigate
    through and access different parts of the game.

\subsection*{Class Definitions}
    \begin{itemize}
        \item \mintinline{python}{RenderedBase} will implement the basic
        interface for any object which is rendered on the screen.
        It provides a way of getting the next image in an sequence
        for rendering.
        \item \mintinline{python}{ActorBase} will describe the interface
        for actors.  All objects which are rendered to the screen and
        perform some action (the user and the enemies) will be subclasses
        of this class.
        \item \mintinline{python}{MainMenu} will allow the user to access
        the main game and to view information such as the high scores
        across sessions through options on a menu, such as New Game and
        High Scores.
    \end{itemize}

\subsection*{Pseudocode}
    % Pseudo-code for Base Classes (Terrence's Portion)
    \inputminted{python}{../assets/base_classes.py}
    %Pseudo-code for Menus (Keefer's Portion)
    \inputminted{python}{../assets/menus.py}

\subsection*{Data File Specifications}
    The data file for LazerBlast will consist of a
    pickled state of all high scores, and, possibly,
    of the current game state (if a user wants to
    continue a longer game.)  The format itself is
    decided by Python, and will represent the internal
    python objects.
