% Detailed Design Specifications
%
% Interface specifications
% Class definitions
% Pseudocode
% Data file specifications if applicable
\subsection*{Interface Specifications}
    The classes \mintinline{python}{RenderedBase} and
    \mintinline{python}{ActorBase} will form a basis for
    the API for LazerBlast.  They describe all moving
    objects, and all actors in the game.  Among their
    most important functions, they will define a method
    to set the current action.  They will also use Python's
    magic methods to get the next frame of their actions.

\subsection*{Class Definitions}
    \begin{itemize}
        \item \mintinline{python}{RenderedBase} will implement the basic
        interface for any object which is rendered on the screen.
        It provides a way of getting the next image in an sequence
        for rendering.
        \item \mintinline{python}{ActorBase} will describe the interface
        for actors.  All objects which are rendered to the screen and
        perform some action (the user and the enemies) will be subclasses
        of this class.
    \end{itemize}

\subsection*{Pseudocode}
    % Pseudo-code for Base Classes (Terrence's Portion)
    \inputminted{python}{../assets/base_classes.py}

\subsection*{Data File Specifications}
    The data file for LazerBlast will consist of a
    pickled state of all high scores, and, possibly,
    of the current game state (if a user wants to
    continue a longer game.)  The format itself is
    decided by Python, and will represent the internal
    python objects.
