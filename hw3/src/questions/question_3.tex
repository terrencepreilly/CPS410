% Detailed Design Specifications
%
% Interface specifications
% Class definitions
% Pseudocode
% Data file specifications if applicable
\subsection*{Interface Specifications}
    % Terrence's portion
    The classes \mintinline{python}{RenderedBase} and
    \mintinline{python}{ActorBase} will form a basis for
    the API for LazerBlast.  They describe all moving
    objects, and all actors in the game.  Among their
    most important functions, they will define a method
    to set the current action.  They will also use Python's
    magic methods to get the next frame of their actions.

    % Meagon's portion
    The GameSound class will release a sound when it receives
    a command telling it that a missile has been fired by
    the player or an enemy ship.  The sound will then be transmitted
    through the game for the player to hear.  The ShipProjectile
    class will contain a method which will create the missile
    object, another method which will assign certain characteristics
    to the missile object such as whether it’s a enemy missile or a
    player missile, and if a player missile then the object will be
    assigned a color based on a command received from the player
    telling the program which color the missile should be. The
    object will then be returned to the class which had called the
    ShipProjectile class in order to retrieve a missile object.

\subsection*{Class Definitions}
    \begin{itemize}
        \item \mintinline{python}{RenderedBase} will implement the basic
        interface for any object which is rendered on the screen.
        It provides a way of getting the next image in an sequence
        for rendering.
        \item \mintinline{python}{ActorBase} will describe the interface
        for actors.  All objects which are rendered to the screen and
        perform some action (the user and the enemies) will be subclasses
        of this class.
        \item \mintinline{python}{GameSound} class will implement
        the sound aspects of the game which will provide the
        missile objects with sound every time they are fired and
        when they have made contact with the target
        ship.
        \item \mintinline{python}{ShipProjectile} class will create
        a new missile object when called and assign certain
        characteristics to that object based on information
        received. Once the object has been created, it will then
        be returned to the class which had called for it to be created.
    \end{itemize}

\subsection*{Pseudocode}
    % Pseudo-code for Base Classes (Terrence's Portion)
    \inputminted{python}{../assets/base_classes.py}

    % Pseudo-code for Sound and Projectiles (Meagon's Portion)
    \inputminted{python}{../assets/sounds_and_projectiles.py}

\subsection*{Data File Specifications}
    One data file for LazerBlast will consist of a
    pickled state of all high scores, and, possibly,
    of the current game state (if a user wants to
    continue a longer game.)  The format itself is
    decided by Python, and will represent the internal
    python objects.

    Another data file for LazerBlast will consist of input data received
    from the player in order to allow the program to determine when
    a sound file needs to be accessed and played, as well as when to
    create a missile object and what characteristics to assign to
    that newly created object.
