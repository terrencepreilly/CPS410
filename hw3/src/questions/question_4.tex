% Test Plan Specifications
%
% Performance tests
% Stress tests
% Functional tests
In order to keep track of our tests we will be using a chart to
track the test number, the test data, the test purpose, our expected
results, and whether or not the test was successful. Below is a
sample of the chart which we will be using to track our tests and
their results.

% Insert Table
\begin{center}
    \begin{tabular}{| p{0.125\linewidth} | p{0.125\linewidth} | p{0.125\linewidth} |
                      p{0.125\linewidth} | p{0.125\linewidth} | p{0.125\linewidth} |}
    \hline
          Test No.
        & Test Data
        & Reason
        & Expected Results
        & Success
        & Comments \\
    \hline
          1
        & Enter temporary unlimited health for player
        & To test hit on enemy player without taking damage.
        & Enemy player will take hit and be terminated while colors match appropriately.
        & Yes
        & Test was able to be completed successfully.  \\
    \hline
          2
        & Replace algorithm related to enemy ships.
        & Trying to make the path our enemy ships follow more efficient.
        & Enemy ships will follow same path using a more efficient algorithm.
        & Enemy ships getting stuck when program is running with new
          algorithm. Algorithm needs altered and further testing required.
        & \\
    \hline
    \end{tabular}
\end{center}


Tests relating to the syntax of our program shall be performed
through the use of both the unit test framework provided through
Python, as well as a code review process which will involve
developers reviewing each other’s code and offering any advice on
possible found issues.

\subsection*{Performance Tests}
    The performance tests will be documented within our test
    documentation sheet where we give the test data being used to
    test the performance, the reason which will specify what exactly
    relating to the performance is being tested, our expected
    results, whether or not the test was successful, and any
    additional comments which the team member who performed the
    tests feels should be brought to the developers and users
    attention.  Some of the steps that will be taken prior to
    performance tests are to make sure that all of the correct
    software has been installed so that the game may perform to the
    best of its ability.

    One of the main things which our performance tests will include
    is the speed and smoothness of our programs flow since python
    games created and ran within Pygame tend to have issues with
    their speed. A lot of the issues relating to the performance
    test are often related to the syntax used in developing the
    game. Steps to avoid these being issues will include using
    certain modules, libraries, or even just replacing the syntax
    with something which may help to scientifically improve the
    speed and overall performance of our game.

\subsection*{Stress Tests}
    The stress tests which will be performed in relation to our game
    will include increasing the number of enemy ships to see how
    many ships the program may be able to handle at once. This test
    will allow our developers to determine what a good number of
    enemy ships may be in order to provide the user with a smooth
    flowing game as well as a challenge. This stress test will also
    help developers in determining how well the path finding
    algorithm will work for the chosen number of enemy ships at any
    one time. Another test to be performed under the stress test
    phase will be the intensity of the boss enemy ship and what all
    factors may play in the fight between the player and the boss
    ship. Once again this test will allow developers to provide the
    players with a smooth flowing program as well as a challenging game.

\subsection*{Functional Tests}
    Once both the performance tests and stress tests have helped in
    determining syntax, libraries, modules, and any other relating
    factors to be used within the program, then functional tests
    shall be performed by going through the game and focusing on
    each phase and detail. Once our group has performed Alpha
    testing of the game and have made sure that the game is
    functioning as planned then the game shall be released to a
    select group of individuals for Beta testing so that we may
    receive feedback on any functionality issues which we missed.
    During both the Alpha and Beta testing processes, documentation
    shall take place in reporting each issue, the proposed solution,
    testing of the program with the solution in place and a
    recording of the results. Each reported issues shall be dealt
    with and the process of Alpha and Beta testing will be repeated.
    Once our group and those individuals chosen to participate in
    the Beta testing process are satisfied with the games
    functionality, we will consider all tests complete.
