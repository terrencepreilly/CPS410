\subsection*{Overview of an Existing System}
% - Describe the existing system from which you are to improve for your
%  proposed research project.  Include its system architectures where
%  appropriate
% - Stress a specific area of the existing system where you are to focus on and
%  indicate the value of your research and its contribution to the field.
Galaga's gameplay consists of managing the movement and firing of a
ship in order to avoid incoming projectiles whilst scoring hits on
enemy ships by firing their own. The crux of this system's
architecture as well as our own is the ship class. The fields for
the ship class include both a rect used for determining the position
of the a ship and an image to represent it. In addition to a method
for movement, a method for firing exists, which creates a projectile
object. The projectile class posesses the same fields as the ship,
but lacks the method for firing.

A significant portion of the user's attention is drawn to the paths
of moving projectiles. Our improvement upon this system is to do
away with the projectile in favor of adding a weapon class which
possesses a method to instantaneouslyg draw a laser beam on the
screen and register damage to the enemies hit by it.g This instant
damage allows the user to hit fast-moving targets by virtue ofg
reaction time rather than by prediction projectile and ship
movements. Tog introduce extra challenge for the user, both the ship
and weapon class have color as a property. When registering damage,
a weapon only damages a ship of matching color. A ship posesses a
list of weapons and a method to switch the active weapon and fire
it. This addition creates an imperative to match their ship's active
weapon to the color of the enemy they are firing upon.

\subsection*{The Proposed System}
% Overview
Our proposed system is "Lazer Blast", an arcade-style shooter (like
\href{https://en.wikipedia.org/wiki/Galaga}{Galaga}), which includes
aspects of color matching and pattern matching similar to Simon.

\subsubsection*{Game-play}
The user pilots a small ship against hoards of enemies.  Each enemy
has a given color.  The player must use a similar-colored laser to
destroy that enemy.  For example, an enemy which is green much be
shot with a green laser.  A red laser will have no effect on a green enemy.
Waves of enemies come in patterns: an astute fighter will be able to
predict the next wave from the previous, and will have a significant
advantage over the enemy.

Since the player cannot be expected to handle the copious masses
of colorful enemies immediately, the difficulty of the game increases
as the player progresses, starting with a single color and slowly
adding other colors and increasing the speed.
% TODO: Include back story?

\subsubsection*{System Architecture}
Lazer Blast was designed using Object-Oriented Design, and implemented
with an Object-Oriented Approach using Python and the game library,
\href{http://pygame.org/news}{pygame}.  Python was choosen for the
ability to rapidly prototype, as well as its widespread use among
the open-source community.  Object-Oriented Programming is well
adapted to Python: Python, while being a multi-paradigm language,
has a strong tendency towards Object-Oriented Programming.  The
framework, pygame, was chosen due to its relative simplicity and due
to its relative impartiallity towards individual implementation choices.

In the Object-Oriented Design, all objects which are drawn to the screen
are subclasses of the \mintinline{python}{RenderedBase} class.  All
objects which perform some action subclass the \mintinline{python}{ActorBase}
class.  For example, obstacles will be subclasses of
\mintinline{python}{RenderedBase}, but all enemies will be subclasses of
\mintinline{python}{ActorBase} and of \mintinline{python}{RenderedBase}.
For the initial version of the game, there is only one obstacle,
\mintinline{python}{Obstacle}, and only two types of actors:
\mintinline{python}{Player} and \mintinline{python}{Enemy}.

% Class diagram of RenderedBase, ActorBase, Obstacle, Enemy, and Player.

\subsubsection*{Algorithms}

% Algorithms
Gratuitous use of Python's magic method, \mintinline{python}{__next__}
was made to implement all sequential algorithms that were necessary.
In this way, each object which maintained a state through a sequence
acts as a generator.  This eases the complexity of logic, and makes
maintaining state easier. (As the individual components exhibit looser
coupling, and stronger cohesion.)  For example, the images to be blitted
to the screen for a given actor are defined as follows:

\inputminted{python}{../code/RenderedBase.py}

The \mintedinline{python}{GameSound} class will implement the sound
aspects of the game.  It will provide the lasers with sound every
time they are activated, will play a crashing sound when a specific
colored laser hits the matching colored target and will provide a
sound specific to the player losing or winning the game.

\inputminted{python}{../code/GameSound.py}

The \mintidinline{python}{LaserStrike} class will display a laser
object of a specific color when a certain key is selected. If the
laser is of the same color as the enemy target and makes contact,
then this class will call the \mintedinline{python}{GameSound} class
in order to implement the crashing sound that will return from a
direct hit. This class will also call the \mintedinline{python}{Enemy}
class in order to deduct health upon a hit.

\inputminted{python}{../code/LaserStrike.py}
