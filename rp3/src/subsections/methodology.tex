\subsection*{Overview of an Existing System}
% - Describe the existing system from which you are to improve for your
%  proposed research project.  Include its system architectures where
%  appropriate
% - Stress a specific area of the existing system where you are to focus on and
%  indicate the value of your research and its contribution to the field.

\subsection*{The Proposed System}
% Overview
Our proposed system is "Lazer Blast", an arcade-style shooter (like
\href{https://en.wikipedia.org/wiki/Galaga}{Galaga}), which includes
aspects of color matching and pattern matching similar to Simon.

\subsubsection*{Game-play}
The user pilots a small ship against hoards of enemies.  Each enemy
has a given color.  The player must use a similar-colored laser to
destroy that enemy.  For example, an enemy which is green much be
shot with a green laser.  A red laser will have no effect on a green enemy.
Waves of enemies come in patterns: an astute fighter will be able to
predict the next wave from the previous, and will have a significant
advantage over the enemy.

Since the player cannot be expected to handle the copious masses
of colorful enemies immediately, the difficulty of the game increases
as the player progresses, starting with a single color and slowly
adding other colors and increasing the speed.
% TODO: Include back story?

\subsubsection*{System Architecture}
Lazer Blast was designed using Object-Oriented Design, and implemented
with an Object-Oriented Approach using Python and the game library,
\href{http://pygame.org/news}{pygame}.  Python was choosen for the
ability to rapidly prototype, as well as its widespread use among
the open-source community.  Object-Oriented Programming is well
adapted to Python: Python, while being a multi-paradigm language,
has a strong tendency towards Object-Oriented Programming.  The
framework, pygame, was chosen due to its relative simplicity and due
to its relative impartiallity towards individual implementation choices.

In the Object-Oriented Design, all objects which are drawn to the screen
are subclasses of the \mintinline{python}{RenderedBase} class.  All
objects which perform some action subclass the \mintinline{python}{ActorBase}
class.  For example, obstacles will be subclasses of
\mintinline{python}{RenderedBase}, but all enemies will be subclasses of
\mintinline{python}{ActorBase} and of \mintinline{python}{RenderedBase}.
For the initial version of the game, there is only one obstacle,
\mintinline{python}{Obstacle}, and only two types of actors:
\mintinline{python}{Player} and \mintinline{python}{Enemy}.

% Class diagram of RenderedBase, ActorBase, Obstacle, Enemy, and Player.

\subsubsection*{Algorithms}

% Algorithms
Gratuitous use of Python's magic method, \mintinline{python}{__next__}
was made to implement all sequential algorithms that were necessary.
In this way, each object which maintained a state through a sequence
acts as a generator.  This eases the complexity of logic, and makes
maintaining state easier. (As the individual components exhibit looser
coupling, and stronger cohesion.)  For example, the images to be blitted
to the screen for a given actor are defined as follows:

% Include RenderedBase.__next__ code.
