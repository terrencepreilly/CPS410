% Chapter 6	Exception Handling
\subsection*{Exception handling}
    There are a number of exceptions that may occur and disrupt the
    experience of the user.

\subsection*{Collision detection errors}
    The users in-game character should be destroyed by colliding
    with in-game enemies and laser beams. Should the game fail to
    detect collisions, the in-game enemies or player may become
    invulnerable. Should the collision detection produce
    false-positives, the user may lose the game without failing to
    dodge enemies and laser beams.

\subsection*{Movement errors}
    The user's in-game character should remain on screen at all
    times. If the code responsible for this task should fail, the
    user's session will be disrupted.

\subsection*{Audio exceptions}
    If a user's audio drivers are missing or corrupted, the game's
    sound will not function properly. Most properly-installed
    operating systems should not encounter this exception.

\subsection*{Frame-Rate Exceptions}
    The game should update at a constant rate of rendered frames per
    second. If the code responsible fails, the game may perform at
    an accelerated pace on faster computers, making the game far
    more difficult to play than on slower computers.

\subsection*{Graphics Scaling exceptions}
    Depending on the resolution of the user's computer, the game
    will have to scale its graphics to fit the screen. If there is
    an error in scaling, the size and position of in-game objects
    may become inconsistent between resolutions. For this reason the
    game should be tested on a selection of common resolutions to
    ensure this does not occur.

\subsection*{Invalid Inputs}
    Whilst the user controls their character via keyboard input it
    is possible they may press the wrong keys. Whilst the game
    itself should not behave poorly even when an invalid key is
    pressed, some operating systems such as Ubuntu and Windows may
    change focus from the game's window, disrupting the user's session.

\subsection*{Unknown Inputs}
    Upon the first use of the game, a user may not know which inputs
    correspond to which actions in the game. If this information is
    not made available to the user anywhere in the game, the user
    may learn the correct inputs by trial and error, or become
    frustrated and quit. Therefore the inputs should be made
    available to the player before they play for the first time.
