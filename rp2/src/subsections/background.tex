% Terrence's Section
One of the first electronic, microcomputer-controlled games, \textit{Simon},
has also been one of the highest selling games of all time.  \textit{Simon}
was invented in the 1970s by Ralph Baer and Howard J. Morrison, and
was marketed through Hasbro.  \textit{Simon} greatly influenced later games,
especially music and rhythm games such as \textit{Dance Dance Revolution}.
\citep{austin2016}.

Part of \textit{Simon}'s popularity came from its variable difficulty.  Baer
created \textit{Simon} with a toggle switch allowing players to make the game
more difficult as they progressed.  \textit{Simon}'s simplistic colors and low
cost also were important factors in its success.  Finally, part of
\textit{Simon}'s success also lay in the notion that gameplay could improve
cognitive abilities.  This belief is not entirely unfounded.

% Meagon's section
In \textit{Cognitive Benefits of Computer Games for Older Adults},
E. Zelinski and R. Reyes discuss how studies have demonstrated that playing video
games help with reducing or delaying cognitive diseases such as
Alzheimer’s in adults.  However, unlike \textit{Simon}, the games which appear
to have the largest affect in producing benefits are digital action games.
The authors further point out that adaptive difficulty not only makes
games more enjoyable, but also make patients more likely to continue
playing them.

Many of the games from the era of \textit{Simon} are approaching an age
where they will be susceptible to cognitive decline.  Althogh it
may seem an intractable group for the gaming industry, 49\% of gamers
are 18-49 years old, with an average age of 35, and 26\% of gamers are
over 50 years of age \citep{zelinski2009}.
