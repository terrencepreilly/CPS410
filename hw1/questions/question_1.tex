% EXAMPLE
% 1.  Problem Domain
% The Office of Information Technology provides, among other
% services, the service of creating registration forms for events
% such as conferences. This form collects information about the
% registrant, provides a mechanism to collect a fee for such
% registration, and gives the organizer a way to manage the booking
% list (editing information, updating payments, etc.). Creating a
% new form for each new event is time consuming, and prevents OIT
% resources from committing time to more non-trivial projects.
% Therefore, OIT would like to have a web application developed that
% can automate the majority of this process to free up developer
% time.

Pygame is an open-source framework for creating games.  Games written
with Pygame generally run on a full desktop environment, but can be
configured (through libraries such as \code{pygame\_sdl2}) to run on Android
devices.

However, Pygame has not seen widespread adoption.  One area in which
Pygame could excel is in games which are suitable for children.  A simple,
easy-to-learn, and fast-paced game could easily be written for Pygame
to help increase its fanbase.
