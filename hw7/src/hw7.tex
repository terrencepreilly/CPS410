\documentclass[a4paper,12pt]{article}

\usepackage{amsmath}
\usepackage{graphicx}
\usepackage{minted}
\usepackage[colorlinks=true]{hyperref}
\setkeys{Gin}{width=\linewidth}
\usepackage[margin=1.0in]{geometry}

\author{%
    Reilly, Terrence\
    \and
    Justice, James\
    \and
    Goodall, Brad\
    \and
    Maresh, Keefer\
    \and
    Gleason, Meagon\
}

\author{Terrence Reilly}
\title{Phase VII}

\begin{document}
    \maketitle

    \section*{Objectives}
With our project we aimed to create a spiritual successor to top-down shooters
similar to Galaga. In this game we aimed to add challenge by implementing a
color-matching game mechanic. Lastly, we sought to follow standard practices for
Python modules and make the game extensible for later developers.

    \section*{Implementation Language}
Our language of choice was Python with the PyGame library. We chose Python because
of its widespread use, availability of open-source components on PyPl, and the
ease with which modules can be packaged and released.

    \section*{Lines of Code and Functions}
In total LazerBlast contains 1120 lines of code. These lines are broken up into
84 successfully implemented functions, each around 13 lines on average.

    \section*{Results}
Overall, our project was successful. We have produced a finished product.
We have also implemented a color-matching game mechanic. Finally, we have
followed standard practices for python modules including tests, encapsulation
og logic in a clear hierarchy of classes, and the ability to install this module
using pip.



\end{document}
