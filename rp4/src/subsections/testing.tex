\subsubsection*{Implementation and Testing}
% 3.2.4  Implement the algorithms
%
% 3.2.5  Test your program
%
% 3.2.6  Evaluate the results of the test

% Terrence's section
%Implementation
The base classes for \textit{Lazer Blast} were completed using
the Test-Driven Development paradigm.  A total of 12 unit tests
were written to implement the base classes, giving a solid foundation
for the rest of the program.  The unit tests in question used
a stubbed subclass of \mintinline{python}{RenderedBase} to allow unit
testing:

\inputminted[baselinestretch=1]{python}{../code/testing_mock_subclass.py}

The stubbed class and examples from unittests also helped develop
a contract for creating subclasses at a later point, and so were
successfull.

For instance, a unittest describing that actions are required provided
an example for the later implementation of the actor class:

\inputminted[baselinestretch=1]{python}{../code/testing_actions_required.py}

Difficulties in testing highly components were encoundered.  For
example, when testing the render method for the
\mintinline{python}{RenderedBase} class, it would be expected to rely
upon \mintinline{python}{pygame.draw.rect}.  That is, when rendering to
the screen, it would be expected to render as rectangle.  To test this,
it was necessary to using Python's standard \mintinline{python}{mock}
library:

\inputminted[baselinestretch=1]{python}{../code/testing_mocking.py}


% Testing
% Evaluating

% Meagon's section
The \mintinline{python}{GameSound} class is called when sounds are
to be implemented during specific points within the game. This class
works by submitting a string to the class through either a parameter
or assigning the string within the class. For testing purposes, we
have used the string “loss” which activates two sounds, one of a
pilot saying “mayday, mayday” and another sound of the ship
crashing.  During the creation of this class, we ran into issues
with successful testing as at first the sound did not play through.
After some research and modification of the code, we quickly
realized that the issue causing this was that we had forgotten to
call \mintinline{python}{pygame.mixer.init()}, which initializes the
Pygame sound and allows us to use the built in modules in order to
play sound files being loaded in. After adding this call, we had
once again attempted to run this class, using the pre-assigned
string to activate a certain sound file. While the sound affects did
come back this time and were correct, there was still an issue with
the time gap between the first sound file of the pilot speaking and
the second sound file of the ship crashing. This issue is one which
we had expected to occur prior to testing and the reason which we
chose the string, “loss”. We knew that this string had two sound
files linked to it instead of one and that these sound files would
both need to play and within a certain timeframe of one another.  We
wanted the first sound file to be cut off and the second sound file
to start playing right before the first would end should it play all
the way through. In order to handle this issue, we called
\mintinline{python}{pygame.time.wait()} and had to adjust the number
of milliseconds that we enter into the parameter so that our first
sound file would had enough time to play through the piece which we
wanted to use and so that our second sound file would not start
playing until the right moment. The time frame we chose gives the
player a sense that their pilot is calling for help as they are
going down but crashes during the help call and therefore indicates
a loss for the player. After some back and forth testing, entering a
different timeframe within the \mintinline{python}{pygame.time.wait()}
module, we were able to finally find the perfect timeframe so that
we could accomplish the sound affects we desired for a loss within
the game.
