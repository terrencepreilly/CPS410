\subsubsection*{Implementation and Testing}
% 3.2.4  Implement the algorithms
%
% 3.2.5  Test your program
%
% 3.2.6  Evaluate the results of the test

% Terrence's section
%Implementation
The base classes for \textit{Lazer Blast} were completed using
the Test-Driven Development paradigm.  A total of 12 unit tests
were written to implement the base classes, giving a solid foundation
for the rest of the program.  The unit tests in question used
a stubbed subclass of \mintinline{python}{RenderedBase} to allow unit
testing:

\inputminted[baselinestretch=1]{python}{../code/testing_mock_subclass.py}

The stubbed class and examples from unittests also helped develop
a contract for creating subclasses at a later point, and so were
successfull.

For instance, a unittest describing that actions are required provided
an example for the later implementation of the actor class:

\inputminted[baselinestretch=1]{python}{../code/testing_actions_required.py}

Difficulties in testing highly components were encoundered.  For
example, when testing the render method for the
\mintinline{python}{RenderedBase} class, it would be expected to rely
upon \mintinline{python}{pygame.draw.rect}.  That is, when rendering to
the screen, it would be expected to render as rectangle.  To test this,
it was necessary to using Python's standard \mintinline{python}{mock}
library:

\inputminted[baselinestretch=1]{python}{../code/testing_mocking.py}


% Testing
% Evaluating

% Meagon's section
The \mintinline{python}{GameSound} class is called when sounds are
to be implemented during specific points within the game. This class
works by submitting a string to the class through either a parameter
or assigning the string within the class. For testing purposes, we
have used the string “loss” which activates two sounds, one of a
pilot saying “mayday, mayday” and another sound of the ship
crashing.  During the creation of this class, we ran into issues
with successful testing as at first the sound did not play through.
After some research and modification of the code, we quickly
realized that the issue causing this was that we had forgotten to
call \mintinline{python}{pygame.mixer.init()}, which initializes the
Pygame sound and allows us to use the built in modules in order to
play sound files being loaded in. After adding this call, we had
once again attempted to run this class, using the pre-assigned
string to activate a certain sound file. While the sound affects did
come back this time and were correct, there was still an issue with
the time gap between the first sound file of the pilot speaking and
the second sound file of the ship crashing. This issue is one which
we had expected to occur prior to testing and the reason which we
chose the string, “loss”. We knew that this string had two sound
files linked to it instead of one and that these sound files would
both need to play and within a certain timeframe of one another.  We
wanted the first sound file to be cut off and the second sound file
to start playing right before the first would end should it play all
the way through. In order to handle this issue, we called
\mintinline{python}{pygame.time.wait()} and had to adjust the number
of milliseconds that we enter into the parameter so that our first
sound file would had enough time to play through the piece which we
wanted to use and so that our second sound file would not start
playing until the right moment. The time frame we chose gives the
player a sense that their pilot is calling for help as they are
going down but crashes during the help call and therefore indicates
a loss for the player. After some back and forth testing, entering a
different timeframe within the \mintinline{python}{pygame.time.wait()}
module, we were able to finally find the perfect timeframe so that
we could accomplish the sound affects we desired for a loss within
the game.

The \mintinline{python}{LazerStrike} function is one which activates
the image of a particular colored laser striking as the correct keys
are pressed.  The way this function works is the player hits a
certain key and based on the key they hit, the laser will display in
either red, green, blue, or yellow depending on the key hit. These
colors are ones which will be used for our enemy ships as well,
which is why the key selection is important because if the player
selects the wrong key then the colors will mismatch between the
laser and the enemy ship and therefore the players attempted hit
will miss.  Another color which has been added to this function is
white, this color is called anytime a key within an assigned event
is clicked which will automatically be calculated as a miss for the
player.  When testing this function, there were several errors which
we had run into with the first being that the laser line did not
want to show up on the screen at all because we had not used the
\mintinline{python}{pygame.KEYDOWN} module which is used to listen
for when the player has a key pressed down. Once this was entered
into our function prior to key events being assigned, then we
realized that we had to adjust the position of the laser as it was
going all the way from the top of our screen to the bottom. To fix
this we simply played around with the start and end positions
entered into the drawing module which gave us a better overall
position of our lasers display. The positioning of our laser drawing
is likely something else which will need adjusted again later as we
get our final positions of the players ship and the enemies’ ships
as we want our laser to look as if it is coming from our ship to the
enemy ships.  One way which we could do this is by assigning the
players ships position to the \mintinline{python}{start_pos}
variable and the targeted enemy ships position to the
\mintinline{python}{end_pos} variable which we can pass to our laser
drawing module.

During our testing of this function we had also realized that we
were calling the wrong module when checking for which key was being
held down as we were using the event.type module instead of the
event.key module to assign keys being listened for. Once we had
assigned the correct module to each key listening event statement,
then our keys correctly called the color which we had linked them
to. Our next problem was that the laser did not disappear after
being called, it stayed on the screen and simply changed color. The
ideal fix for this would be to completely remove our laser image,
which we’re hoping to implement prior to our final release of the
game. Unfortunately, we have yet to accomplish this fix but we were
able to apply a fix where the laser would only show up as a visual
to the player when they were holding down the assigned key. The way
in which we were able to fix this was by implementing the
\mintinline{python}{pygame.KEYUP} module which listens for when no
keys are being used.  At this point the laser color is assigned to
match background color of our game. We can then make it so that this
color has no effect, should this be our ultimate fix. Assigning the
color of the laser to the same color of the background while no keys
are being pressed, indicates to the player that the laser is not
active. If there are multiple colors being used almost back to back,
there would still be that flash in-between keystrokes of a laser not
being activated which will provide the player with the sense of the
lasers being changed from one to the next instead of showing a
constant active laser which only changes in color.
